\documentclass[11pt, letterpaper]{article}

% --- PACKAGES ---
\usepackage[table]{xcolor}
\usepackage[utf8]{inputenc}
\usepackage{pdfpages}      % To attach the scanned PDFs
\usepackage[hidelinks]{hyperref} % Clickable TOC in the PDF
\usepackage{fancyhdr}      % For page numbering and section headers
\usepackage[margin=1in]{geometry}
\usepackage{longtable}     % For tables that span multiple pages
\usepackage{booktabs}      % Professional table borders
\usepackage{array}
\usepackage{helvet}
\usepackage{graphicx}
\usepackage{float}
\usepackage{caption}
\usepackage{etoc} % Allows for local tables of contents
\usepackage{ifthen}

% --- FONT SETTING ---
\renewcommand{\familydefault}{\sfdefault}

\setlength{\parindent}{0pt}

% --- SMART FOOTER CONFIGURATION ---
\pagestyle{fancy}
\fancyhf{} 

% Left Footer: Logic to toggle the dot
\fancyfoot[L]{%
    \small \textbf{\nouppercase{\leftmark}}% Always print Part Name
    \ifthenelse{\equal{\rightmark}{}}% IF RightMark is empty...
        {}% THEN do nothing
        { $\cdot$ \nouppercase{\rightmark}}% ELSE print " dot DocumentName"
}

% Right Footer: Page Number
\fancyfoot[R]{\small Page \thepage}           
\renewcommand{\headrulewidth}{0pt}  
\renewcommand{\footrulewidth}{0.4pt}

% --- CUSTOM COMMANDS ---

% The Sub-header for narratives
\newcommand{\sectiontitle}[1]{
    \vspace{0.5cm}
    \noindent\textbf{\large #1} 
    \vspace{0.2cm}
    \hrule
    \vspace{0.3cm}
}

% A macro to create a "Part" Divider Page
% It creates a new page, centers the Title heavily, and adds it to TOC
\makeatletter % <--- REQUIRED: Allows us to use the internal \l@ commands



% THE NEW "SMART" BINDER PART
% 1. Starts the Section
% 2. Prints the Title Page
% 3. Generates an Automatic List (TOC Style with Page Numbers)
\newcommand{\BinderPart}[2]{%
    \clearpage
    \phantomsection
    \refstepcounter{section} 
    \addcontentsline{toc}{section}{#1: #2} 

    % UPDATED: Set the Left Mark (Part), Clear the Right Mark (Doc)
    \markboth{#1: #2}{} % This ensures that on the Part Cover Page, the footer just says "PART 1: Application $\cdot$" without repeating itself.
    
    % The Visual Cover Page
    \vspace*{2cm} 
    \begin{center}
        {\Huge \textbf{#1}} \\[0.5cm] 
        {\Huge #2} 
    \end{center}
    \vspace*{1.5cm}
    
    % --- ETOC CONFIGURATION (Mini TOC) ---
    
    % 1. Go deep enough to see Sub-Subsections (Exhibits/Checks)
    \etocsetnexttocdepth{subsubsection}
    
    % 2. Silence the 'Section' (Part Title) so it doesn't duplicate
    \etocsetstyle{section}{}{}{}{} 
    
    % 3. Use standard LaTeX TOC styling for subsections & subsubsections
    % We replace the deprecated command with \l@subsection{#1}{#2}
    \etocsetstyle{subsection}
        {}                            % Start
        {}                            % Prefix
        {\l@subsection{\etocname}{\etocpage}} % Content (Standard dotted line)
        {}                            % Suffix

    \etocsetstyle{subsubsection}
        {}
        {}
        {\l@subsubsection{\etocname}{\etocpage}}
        {}
        
    % 4. Print the local list
    \localtableofcontents
    
    \clearpage
}
\makeatother % <--- REQUIRED: Returns @ to normal behavior


\begin{document}
\pagenumbering{arabic} 

\begin{titlepage}
    \centering
    \vspace*{1cm}
    
    % --- LOGO / HEADER (Optional) ---
    % If you have a university or personal letterhead logo, you could put it here.
    % Otherwise, a simple rule looks professional.
    \hrule height 2pt
    \vspace{0.5cm}
    
    % --- TITLE ---
    {\Huge \textbf{MEDICAID APPLICATION}} \\[0.5cm] 
    {\LARGE \textbf{Chronic Care / Nursing Home Coverage}}
    \vspace{0.5cm}
    \hrule height 2pt
    
    \vspace{2.5cm}
    
    % --- APPLICANT INFO ---
    {\Large \textbf{APPLICANT}} \\[0.5cm]
    {\Huge Firstname Lastname} \\[0.5cm]
    {\Large Resident at:\\[0.1cm]
PLACE\\
Rehabilitation and Nursing Center,\\[0.1cm]
City, State}
    
    \vspace{2cm}
    
    % --- REPRESENTATIVE INFO ---
    {\Large \textbf{SUBMITTED BY}} \\[0.5cm]
    {\Large Firstname Lastname (Daughter \& Power of Attorney)} \\[0.2cm]
    {\large address} \\
    {\large City, State zip} \\
    {\large email}
    
    \vspace{2cm}
    
    % --- DATE ---
    {\Large \textbf{DATE OF SUBMISSION}} \\[0.5cm]
    {\Large Month day, year}
    


\end{titlepage}
    
    % --- FRONT MATTER ---
    %\pagenumbering{roman} 
	\setcounter{tocdepth}{2} % Show only Sections and Subsections in the main TOC
\markright{}
    \tableofcontents
    \newpage
    %\pagenumbering{arabic} 
    
    % =================================================================
    % PART 1: APPLICATION & IDENTITY
    % =================================================================
    
    \BinderPart{PART 1}{Forms \& Identity}
    

% 1. Permission to Verify Information
    \phantomsection
    \addcontentsline{toc}{subsection}{Permission to Verify Information}
\markright{Permission to Verify Information}
[The county DSS mailed us a paper to sign.]
  %  \includepdf[pages=-, pagecommand={\thispagestyle{fancy}}, scale=0.9]{permissionVerify.pdf}

\clearpage
    \phantomsection
    \addcontentsline{toc}{subsection}{Permission to Represent Applicant}
\markright{Permission to Represent Applicant}
[The county DSS mailed us a paper to sign.]
 %   \includepdf[pages=-, pagecommand={\thispagestyle{fancy}}, scale=0.9]{PermissionRepresent.pdf}

    \phantomsection
    \addcontentsline{toc}{subsection}{DOH-5148: Authorization for Verification of Resources}
\markright{Authorization for Verification of Resources}
    \includepdf[pages=-, pagecommand={\thispagestyle{fancy}}, scale=0.9]{doh-5148.pdf}

    \phantomsection
    \addcontentsline{toc}{subsection}{DOH-5147: Submission of Application on Behalf of Applicant}
\markright{Submission of Application on Behalf of Applicant}
    \includepdf[pages=-, pagecommand={\thispagestyle{fancy}}, scale=0.9]{doh-5147.pdf}


    % 3. Power of Attorney
    \phantomsection
    \addcontentsline{toc}{subsection}{Power of Attorney [notarized]}
\markright{Power of Attorney}
    \includepdf[pages=-, scale=0.91, pagecommand={\thispagestyle{fancy}}]{POA.pdf}
    

    \phantomsection
    \addcontentsline{toc}{subsection}{DOH-4220: Medicaid application form}
\markright{Application}
    \includepdf[pages=-,  pagecommand={\thispagestyle{fancy}}]{doh-4220.pdf}


    % 3. Supplement A
    \phantomsection
    \addcontentsline{toc}{subsection}{DOH-5178: Supplement A}
\markright{Supplement A}
    \includepdf[pages=-, scale=0.9, pagecommand={\thispagestyle{fancy}}]{doh-5178a-supplementA-empty.pdf}
    
    \phantomsection
    \addcontentsline{toc}{subsection}{Identity Documents}
\markright{Passport}
\sectiontitle{Passport}
\noindent
[ passport ]
\newpage
%\includegraphics[height=0.95\textheight]{passport}
\markright{Identity documents}
[ birth certificate, social security card, driver license ]
    %\includepdf[pages=-, pagecommand={\thispagestyle{fancy}}]{identitydocuments.pdf}
\newpage
    \phantomsection
    \addcontentsline{toc}{subsection}{Health insurance}
\markright{Health insurance}
 [insurance cards, scanned]
   % \includepdf[pages=-, pagecommand={\thispagestyle{fancy}}]{insurancecards.pdf}

\newpage
\markright{Medicare}
   % \includepdf[pages=-, pagecommand={\thispagestyle{fancy}}]{MedicareCard.pdf}


\sectiontitle{Medicare insurance coverage report}

\noindent
\hspace{-.5in}
 [printout from Medicare 1]
    %\includegraphics[scale=0.7]{MedicareInfo1.pdf} 

\noindent
\hspace{-.5in}
[printout from Medicare 2]
    %\includegraphics[scale=0.7]{MedicareInfo2.pdf}

[Medicare Premiums document]
    %\includepdf[pages=-, scale=.97, pagecommand={\thispagestyle{fancy}}]{MedicarePremiums.pdf}


    \newpage
    \phantomsection
    \addcontentsline{toc}{subsection}{Nursing Home Documents (PRI/Admission)}
\markright{Nursing Home Documents}

[Nursing home admission record]

\includepdf[pages=-, scale=.9, pagecommand={\thispagestyle{fancy}}]{doh-694.pdf}
\includepdf[pages=-, scale=.9, pagecommand={\thispagestyle{fancy}}]{LDSS-3559.pdf}



   % \includepdf[pages=-, pagecommand={\thispagestyle{fancy}}, scale=0.87, trim={3.1em 1cm 1em 1em}, clip]{nursing_home_docs.pdf}
    
    

% =================================================================
    % PART 2: INCOME VERIFICATION
    % =================================================================

    \BinderPart{PART 2}{Income Verification}

      \phantomsection
    \addcontentsline{toc}{subsection}{Summary of Monthly Income (2026)}
\markright{Monthly Income}
 
\begin{center}
        \large \textbf{SUMMARY OF MONTHLY INCOME (2026)}
    \end{center}
    \vspace{0.3cm}

    \begin{table}[h!]
        \centering
        \renewcommand{\arraystretch}{1.5} 
        \begin{tabular}{ >{\bfseries}l l r >{\raggedright\arraybackslash}p{6.2cm} }
            \toprule
            Source & \textbf{Document} & \textbf{Gross Amt} & \textbf{Notes} \\
            \midrule
            
            % ITEM 1: Social Security (2026)
            Social Security & 
            nextyear COLA Notice & 
            \textbf{\$xxxx.xx} & 
            \textbf{Gross Benefit.} Includes nextyear COLA. (Net deposit will be \textbf{\$xxxx.xx} after \$202.90 Medicare premium deduction). \\
            

            % ITEM 2: Bankname3 Annuity
            \rowcolor{gray!10}
            Bankname3 Annuity & 
            Payout Confirmation & 
            \textbf{\$xxx.xx} & 
            \textbf{New Income Stream.} Systematic withdrawal elected [Month year]. \\

\midrule
\textbf{TOTAL} & & \textbf{\$xxxx.xx} & \textbf{Total Anticipated Gross Monthly Income} \\
\bottomrule
        \end{tabular}
    \end{table}

% The expectation is that Medicaid will require almost all of this monthly income to go to the nursing home. This is called NAMI.

    \vspace{1cm}

% --- DOCUMENT LIST ---
    \sectiontitle{Supporting Documentation}
    \begin{itemize}
        \item \textbf{2026 Social Security COLA Notice:} Confirms the new [year] benefit amount. (p.~\pageref{page:COLA})
        \item \textbf{SSA Benefit Verification Letter:} Confirms current active status. (p.~\pageref{page:SSverif})
        \item \textbf{SSA Payment History:} Detailed ledger of disbursements. (p.~\pageref{page:SSPayments})
        \item \textbf{SSA-1099 (year):} Historical income reference. (p.~\pageref{page:SS1099})
        \item \textbf{Bankname3 systematic monthly payout confirmation:} Documentation of Payout Mode election (Month year). (p.~\pageref{page:payout})
    \end{itemize}
    
    \clearpage

    % --- ATTACHMENTS ---

    \phantomsection
    \addcontentsline{toc}{subsection}{Social Services Disability Income (SSDI)}
\markright{SSDI}
    \addcontentsline{toc}{subsubsection}{SSA-4926-SM: SSDI Cost of Living Adjustment (COLA)}
    \label{page:COLA}
[Social Security COLA document]
    %\includepdf[pages=-, scale=0.9, pagecommand={\thispagestyle{fancy}}]{SSCOLA2026.pdf} 
% form SSA-4926-SM

\clearpage
    \phantomsection
    \addcontentsline{toc}{subsubsection}{Social Security Benefit Verification Letter}
    \label{page:SSverif}

   \includepdf[pages=-, scale=0.9, pagecommand={\thispagestyle{fancy}}]{Sample-of-Social-Security-Benefit-Verification-Letter.pdf} 

    \phantomsection
    \addcontentsline{toc}{subsubsection}{Recent SSDI payments}
    \label{page:SSPayments}
[printout of  recent payments from social security to the applicant]
    %\includepdf[pages=-, scale=0.9, pagecommand={\thispagestyle{fancy}}]{SSPayments.pdf} 

    \phantomsection
    \addcontentsline{toc}{subsubsection}{SSA-1099SM: SSDI Benefit Statement for [lastyear]}
    \label{page:SS1099}

    \includepdf[pages=-, scale = 0.9, pagecommand={}]{1099ssa.pdf} 

\clearpage
    \phantomsection
    \addcontentsline{toc}{subsection}{Bankname3 retirement account payout confirmation}
\markright{Bankname3 retirement account payout confirmation}
    \label{page:payout}
[documentation of new Bankname3 payout]
   % \includepdf[pages=-, scale=0.9, pagecommand={\thispagestyle{fancy}}]{Bankname3_payout_confirmation.pdf}


    % =================================================================
    % PART 3: RESOURCES & ASSET HISTORY
    % =================================================================

    \BinderPart{PART 3}{Resources \& Asset History}

\label{sect:part3}


\newpage
% --- RESOURCE COVER PAGE ---
    \phantomsection
    \addcontentsline{toc}{subsection}{Resource Summary}
\markright{Resource summary}
    
    \begin{center}
        \vspace*{1cm}
        {\LARGE \textbf{Resource Summary}} \\[0.5cm]
        {\Large Asset Snapshot as of application date}
    \end{center}

    \vspace{1cm}

    \noindent
    The following table summarizes all liquid assets owned by the applicant. \\
    \textbf{Total Countable Resources are BELOW the Medicaid Resource Limit.}

    \vspace{0.5cm}

    % --- SUMMARY TABLE ---
    \begin{table}[h!]
        \centering
        \renewcommand{\arraystretch}{2.2}
        \setlength{\arrayrulewidth}{1.2pt}
        \begin{tabular}{| p{5.5cm} | >{\raggedleft\arraybackslash}p{4cm} | p{5cm} |}
            \hline
            \rowcolor{gray!25} \textbf{Asset} & \textbf{Current Value} & \textbf{Medicaid Status} \\ \hline
            
            % Item 1
            1. Bankname Checking \newline \scriptsize{(account ending in xxxx)} & 
            \$xxxxx.xx & 
            \textbf{Countable Resource} \\ \hline
            
            % Item 2
            2. Bankname Checking \newline \scriptsize{(account ending in xxxx)} & 
            \$xxx.xx & 
            \textbf{Countable Resource} \\ \hline
            
            % Item 3 
            3. Bankname Retirement \newline \scriptsize{account number} & 
            \textit{(See Note)} & 
            \cellcolor{green!10} \textbf{EXEMPT} \newline \footnotesize{(Converted to Income Stream)} \\ \hline
            
            \hline
            \rowcolor{gray!10} 
            \textbf{TOTAL RESOURCES} & 
            \textbf{\$xxxxx.xx} & 
            \textbf{ELIGIBLE} \newline \footnotesize{(Below Resource Limit)} \\ \hline
        \end{tabular}
    \end{table}

%%% RESOURCE LIMIT: https://nyspltc.health.ny.gov/medicaid/

    \vspace{1cm}

    % --- CRITICAL Bankname3 NOTE ---
    \noindent
    \fcolorbox{black}{white}{
        \parbox{\textwidth}{
            \vspace{0.3cm}
            \centering
\textbf{\Large  {\LARGE \textcolor{red}{!}} IMPORTANT NOTE REGARDING RETIREMENT ACCOUNT} \\
            \vspace{0.3cm}
            \raggedright
            \large
            The retirement account (balance of approx.\ \$XXXXXX) has been placed into \textbf{Systematic Withdrawal (Payout) Mode}.
            
%%% Reference: https://www.health.ny.gov/health_care/medicaid/publications/docs/gia/98ma024.pdf


            \begin{itemize}
                \item Per NYS Medicaid regulations, this asset has been converted to a \textbf{stream of monthly income}.
                \item Please refer to \textbf{Part 2 (Income)} for the verification of the monthly payments.
                \item \textbf{Do NOT count the  balance as a resource.}
            \end{itemize}
            \vspace{0.2cm}
        }
    }

    \vfill
    \begin{center}
        \textit{Detailed 5-year statements and transaction audits\\for the accounts listed above follow in this section.}
    \end{center}

    % --- HEADER ---
    %\begin{center}
        %\LARGE \textbf{ASSET AUDIT SUMMARY} \\
    %\end{center}

\newpage
    \phantomsection
    \addcontentsline{toc}{subsection}{A. Bankname Audit}
\markright{A. Bankname Audit}


    \addcontentsline{toc}{subsubsection}{Bankname Audit Summary}
 % --- HEADER ---
    \begin{center}
        \LARGE \textbf{A. Bankname Checking Audit} \\
        \vspace{0.2cm}
        \large Explanation of Large Transactions ($>$\$2,000)
    \end{center}
    \vspace{0.5cm}

    % --- METADATA TABLE ---
    \noindent
    \begin{tabular}{ @{} l l }
        \textbf{Account Owner:} & Firstname Lastname \\
        \textbf{Account Number:} & Ending in xxxx \\
        \textbf{Audit Period:} & [last five years] \\
    \end{tabular}

    \vspace{0.4cm}

    % --- NARRATIVE ---
    \sectiontitle{Account Narrative}
\noindent
    This account serves as the primary funding source for the applicant's living expenses.
    \begin{itemize}
        \item \textbf{Routine Income:} Monthly Social Security (SSDI) deposits of approx. \$xxxx. (Bankname3 payouts of net \$xxx/month are starting.)
        \item \textbf{Routine Expenses:} Rent, utilities, food, etc.
        \item \textbf{Opening Balance Note:} The opening deposit of \$xxxxx.00 (month year) represents the applicant's accumulated life savings, deposited as cash. (See bank record, p.~\pageref{page:openingBankname}).
    \end{itemize}

    \vspace{0.4cm}


% --- TRANSACTION TABLE ---
    \sectiontitle{Large Transaction Audit ($>$\$2,000)}
    \label{page:keytable}

    {
        \centering
        \renewcommand{\arraystretch}{1.5} 
        \begin{tabular}{ l >{\raggedright\arraybackslash}p{1.6cm}  r >{\raggedright\arraybackslash}p{9cm} }
            \toprule
            \textbf{Date} & \textbf{Type} & \textbf{Amount} & \textbf{Explanation / Source} \\
            \midrule
            
            year-MM-DD & Withdrawal & $-$\$xxxxx.xx & \textbf{Medicaid Trust Funding.} Transfers to NYSARC (d4C) and Funeral Trust. (See Part~\ref{part:trusts}.)\\
            
            \rowcolor{gray!10}
            year-MM-DD & Check \#xxx & $-$\$xxxx.00 & \textbf{NYS Tax (Reimbursement).} Repayment to daughtername Lastname for tax bill. \textit{(See p.~\pageref{sec:taxreimb}).} \\ 
            
            year-MM-DD & E-Check \#xxx & $-$\$xxxx.00 & \textbf{Tax Payment.} US Treasury (IRS) payment for [year] Tax Year. \textit{(See p.~\pageref{sec:checktoIRS}).} \\
            
            \rowcolor{gray!10}
            year-MM-DD & Deposit & \$xxxxx.xx & \textbf{Inheritance.} Bankname3 Life Insurance payout (Est. momfirstname Lastname). \textit{(See p.~\pageref{page:Bankname3let}).} \\
            
            year-MM-DD & Deposit & \$xxxx.00 & \textbf{SSA TREAS 310.} Social Security retroactive/lump sum payment. \\
            
            \rowcolor{gray!10}
            year-MM-DD & Check \#xxx & $-$\$xxxx.00 & \textbf{Transfer to Self.} Funds moved to open Bankname2 FCU account. \textit{(See Bankname2 FCU audit, p.~\pageref{page:Bankname2table}).} \\
            
            year-MM-DD & Deposit & \$xxxxx.00 & \textbf{Opening Deposit.} Cash deposit. \textit{(See bank record, p.~\pageref{page:openingBankname}).} \\
            
            \bottomrule
        \end{tabular}
    }


\iffalse
\clearpage
    \addcontentsline{toc}{subsubsection}{Bankname Balance History Graph}
    % --- GRAPH ---
    \sectiontitle{Balance History}
    \begin{figure}[H]
        \centering
        \includegraphics[width=\textwidth]{Bankname_Balance_Graph_Full.pdf}
    \end{figure}

Balance as of [application date]: \$xxxxx.xx
\fi

    \newpage
    % Exhibit A: Reimbursement (Kept here as it relates to Bankname outflow)
    \phantomsection
    \addcontentsline{toc}{subsubsection}{Check \#xxx: NYS Tax Reimbursement (\$xxxx)}
\label{sec:taxreimb}
    \section*{NYS Tax Reimbursement}
    \noindent
    \textbf{Transaction Summary:} Month day year: Applicant reimbursed daughter \$xxxx via Check \#xxx for a tax bill daughter paid on her behalf.
    \vspace{0.2cm}
    \hrule
    \vspace{0.5cm}

	\label{page:reimb}
    \noindent \textbf{1. THE REIMBURSEMENT (Dec year)}
    \begin{center}
	[check image]
        %\includegraphics[width=0.85\textwidth]{check_xxx_nys_taxes_reimb.pdf}
    \end{center}
    \vspace{0.5cm}

    \noindent \textbf{2. ORIGINAL PAYMENT RECEIPT (Apr year)} \\
    {Official Receipt showing Daughtername Lastname paid \$xxxx to NYS Dept of Taxation using her personal checking account. (next page)}
    \newpage
    \phantomsection
    \addcontentsline{toc}{subsubsection}{IT-370: NYS Tax Extension and Receipt for \$xxxx}
\includepdf[pages=1, scale=0.9, pagecommand={}]{it370.pdf}
[receipt with proof daughter paid state taxes]
    %\includepdf[pages=-, scale=0.9, pagecommand={}]{NYStaxreceipt.pdf}

    \newpage
    \addcontentsline{toc}{subsubsection}{Evidence of State Tax liability: \$xxxx}
    \noindent \textbf{3. THE LIABILITY ([year] Tax Year)} \\
    \textit{Excerpts from [year] NYS Tax Return showing amount owed matches \$xxxx.}
    \vspace{0.5cm}
    \begin{center}
[excerpts from NYS tax return]
        %\includegraphics[width=\textwidth]{NYStax1.png}
    	    \vspace{0.5cm}
	    \textit{(...Skipping to Line 61...)}
	    \vspace{0.5cm}
    	%\includegraphics[width=\textwidth]{NYStax2.png}
	    \vspace{0.5cm}
\noindent
At the time of filing the state taxes, Daughtername had already paid New York State.
	    \vspace{0.5cm}
	%\includegraphics[width=\textwidth]{NYStax3.png}
    \end{center}

    

\begin{center}
    \textit{(Full Tax Return attached in Part~\ref{sect:taxreturns} C, p.~\pageref{page:NYStaxes}).}
\end{center}



\vspace{0.5in}
\noindent \textbf{Conclusion:} This was a dollar-for-dollar reimbursement for a valid expense, not a gift.

    % Bankname PDF Attachments
    \newpage
    \phantomsection
    \addcontentsline{toc}{subsubsection}{Bankname Statements}
[huge packet of bank statements]
   % \includepdf[pages=-, scale=.9, pagecommand={\thispagestyle{fancy}}]{banknameBalance_transactions.pdf}
%\includepdf[pages=-, pagecommand={\thispagestyle{fancy}}, scale = 0.9]{Bankname_statements.pdf}


    % --- SUB-SECTION B: Bankname2 ---
    \newpage
    \phantomsection
    \addcontentsline{toc}{subsection}{B. Bankname2 FCU Audit}
\markright{B. Bankname2 FCU Audit}

 % --- HEADER ---
    \begin{center}
        \LARGE \textbf{B. Bankname2 FCU Audit} \\
    \end{center}
    \vspace{0.5cm}
    
    \noindent
    \begin{tabular}{ @{} l l }
        \textbf{Account Owner:} & Firstname Lastname \\
        \textbf{Financial Institution:} & Bankname2 Federal Credit Union \\
        \textbf{Primary Account:} & \textbf{Checking (Share Draft)} -- Active (Routine expenses) \\
        \textbf{Secondary Account:} & \textbf{Savings (Share)} -- Inactive / Nominal Balance ($\sim$\$1.00) \\
    \end{tabular}
    
    \vspace{0.8cm}
    
    \sectiontitle{Account Narrative}
    The Checking account was opened in Month year for routine expenses for the applicant. It was funded by a transfer from the applicant's primary Bankname account. 
    
    \vspace{0.2cm}
    \noindent
    Activity consists primarily of the initial funding and subsequent routine withdrawals for living expenses (rent checks, utilities, and pharmacy co-pays). There was a recent deposit from Bankname of \$xxx.00 on Month date, year.

    \vspace{0.2cm}
    \noindent
    \textit{Note: The attached statements also reflect a mandatory Share Savings account which has maintained a nominal balance (approx. \$1.00) since opening. This savings account is inactive.}
    
    \vspace{0.6cm}
    
    \sectiontitle{Funding Source (Chain of Custody)}

\label{page:Bankname2table}
    \begin{table}[h!]
        \centering
        \renewcommand{\arraystretch}{1.5}
        \begin{tabular}{ @{} l l r p{7cm} @{} }
            \toprule
            \textbf{Date} & \textbf{Type} & \textbf{Amount} & \textbf{Explanation} \\
            \midrule
            year-MM-DD & Deposit & \$xxxx.00 & \textbf{Opening Transfer.} Funds from Bankname Check \#xxx. (See Bankname audit, p.~\pageref{page:keytable}). \\
            \bottomrule
        \end{tabular}
    \end{table}
    


\iffalse


\newpage

%\begin{minipage}
    \sectiontitle{Balance History (Checking Only)}

    The graph below demonstrates the consistent spend-down of funds for routine expenses. Deposits are transfers from the applicant's Bankname Checking account.

    \begin{figure}[H]
        \centering
        \includegraphics{Bankname2_Balance_Graph.pdf}
    \end{figure}
%\end{minipage}

\fi
    
[printout of Bank2 checking and savings balances from website]
    %\includepdf[pages=-, pagecommand={\thispagestyle{fancy}}]{Bankname2Balance.pdf}

[printout of Bank2 checking recent transactions from website]
   % \includepdf[pages=-, pagecommand={\thispagestyle{fancy}}]{Bankname2_recent_transactions.pdf}
% \includepdf[pages=-, scale=0.9, pagecommand={\thispagestyle{fancy}}]{Bankname2FCUstatements.pdf}

[bank statements, five years]


    % --- SUB-SECTION C: Bankname3 ASSET VIEW ---
    \newpage
    \phantomsection
    \addcontentsline{toc}{subsection}{C. Bankname3   Retirement (Asset History)}
\markright{C. Bankname3   Retirement Audit}

 
    % --- HEADER ---
    \begin{center}
        \LARGE \textbf{Bankname3 Retirement Audit} \\
        \vspace{0.2cm}
        \large 5-Year Medicaid Lookback (Month year -- Month year)
    \end{center}
    \vspace{0.5cm}
    
    % --- METADATA TABLE ---
    \noindent
    \begin{tabular}{ @{} l l }
        \textbf{Account Owner:} & Firstname Lastname \\
        \textbf{Account Numbes:} & xxxxxxxxxxxxx \\
        \textbf{Status:} & \textbf{Payout Mode} (Systematic Withdrawals Active) \\
    \end{tabular}
    
    \vspace{0.8cm}

    % --- NARRATIVE ---
    \sectiontitle{Account Narrative}
    This is a pre-tax retirement annuity. The account has entered "Payout Mode" as of late [year] to generate monthly income for the applicant. 
    \vspace{0.2cm}
    
    \noindent
    \textbf{Compliance Note:} The table below confirms that no lump-sum withdrawals or transfers to third parties occurred during the lookback period. All activity was passive market fluctuation until the recent establishment of systematic payouts (net \$xxx/month).

% calculated with SSA actuarial table to ensure all of the balance is expected to be paid out by the expected month of death.

    \vspace{0.6cm}

    % --- WITHDRAWAL AUDIT ---
    \sectiontitle{Withdrawal Audit (Lookback Period)}
    
    \begin{longtable}{@{} l r c @{}}
        \toprule
        \textbf{Statement Date} & \textbf{Ending Balance} & \textbf{Withdrawals/Transfers} \\
        \midrule
        \endfirsthead
        
        \toprule
        \textbf{Statement Date} & \textbf{Ending Balance} & \textbf{Withdrawals/Transfers} \\
        \midrule
        \endhead
        
        \bottomrule
        \endlastfoot
        
        % DATA ROWS
        year-MM-DD & \$xxxxxx.xx & None \\
        year-MM-DD & \$xxxxxx.xx & None \\
        year-MM-DD & \$xxxxxx.xx & None \\
        year-MM-DD & \$xxxxxx.xx & None \\
        year-MM-DD & \$xxxxxx.xx & None \\
        year-MM-DD & \$xxxxxx.xx & None \\
        year-MM-DD & \$xxxxxx.xx & None \\
        year-MM-DD & \$xxxxxx.xx & None \\
        year-MM-DD & \$xxxxxx.xx & None \\
        year-MM-DD & \$xxxxxx.xx & None \\
        year-MM-DD & \$xxxxxx.xx & None \\
        year-MM-DD & \$xxxxxx.xx & None \\
        year-MM-DD & \$xxxxxx.xx & None \\
        year-MM-DD & \$xxxxxx.xx & None \\
        year-MM-DD & \$xxxxxx.xx & None \\
        year-MM-DD & \$xxxxxx.xx & None \\
        year-MM-DD & \$xxxxxx.xx & None \\
        year-MM-DD & \$xxxxxx.xx & None \\
        year-MM-DD & \$xxxxxx.xx & None \\
        year-MM-DD & \$xxxxxx.xx & None \\
    \end{longtable}
     

\iffalse
    \newpage
    \phantomsection
    \addcontentsline{toc}{subsubsection}{Balance history graph}
\sectiontitle{Balance History}
    \begin{figure}[H]
        \centering
	%\includegraphics{Bankname3_cref_balance_graph.pdf}
    \end{figure}
\fi

\newpage

[webpage printout showing current balance and recent transactions for Bankname3]

 % \includepdf[pages=-, pagecommand={\thispagestyle{fancy}}, scale=0.9]{Bankname3_current_balance.pdf}
    
    \newpage
    \phantomsection
    \addcontentsline{toc}{subsubsection}{Statements}

[huge packet of statements from last 5 years]

  %\includepdf[pages=-, pagecommand={\thispagestyle{fancy}}, scale=0.9]{Bankname3_statements.pdf}


    % =================================================================
    % PART 4: THE SPEND-DOWN (Irrevocable TRUSTS)
    % =================================================================

    \BinderPart{PART 4}{The Spend-Down (Irrevocable Trusts)}

\label{part:trusts}

    \phantomsection
    \addcontentsline{toc}{subsection}{Irrevocable Trusts summary: Final spend-down to eligibility limits}
    \begin{center}
        \LARGE \textbf{EXEMPT ASSETS \& TRUST FUNDING} \\
        \vspace{0.2cm}
        \large Final Spend-Down to Eligibility Limits
    \end{center}
    \vspace{0.5cm}
    
    % --- SUMMARY BOX ---
    \noindent\fbox{%
        \parbox{\textwidth}{%
            \vspace{0.3cm}
            \textbf{Summary of Compliance:}
            \begin{itemize}
                \item Both asset transfers are \textbf{exempt} under NYS Medicaid rules.
                \item \textbf{Funding Source:} Bankname Official Checks (withdrawn Month day, year).
            \end{itemize}
            \vspace{0.2cm}
        }%
    }
    \vspace{0.8cm}
    
    \begin{table}[h!]
    \centering
    \renewcommand{\arraystretch}{1.5}
    \begin{tabular}{@{} p{4cm} p{3.5cm} l p{5cm} @{}}
    \toprule
    \textbf{Trust Entity} & \textbf{Trust Type} & \textbf{Amount} & \textbf{Funding Instrument} \\
    \midrule
    \textbf{NYSARC, Inc.\newline Community Trust} & Pooled Trust & \textbf{\$xxxxx.xx} & Bankname Official Check\newline \#xxxxxxxxxx \\
    \midrule
    \textbf{Funeral Home} & Irrevocable\newline Funeral Trust & \textbf{\$xxxxx.00} & Bankname Official Check\newline \#xxxxxxxxxx \\
    \bottomrule
    \end{tabular}
    \end{table}
    
    \vspace{0.5cm}
    
    \noindent \textbf{Attached Documentation:}
    \begin{enumerate}
        \item \textbf{Exhibit A (NYSARC):} Copy of Funding Check stub and Joinder Agreement.
        \item \textbf{Exhibit B (Funeral):} Copy of Funding Check stub, Itemization, and Trust Contract.
    \end{enumerate}

    \clearpage
    
    % --- NYSARC SECTION ---
        \subsection*{A. NYSARC Pooled Trust (d4C)}
    \addcontentsline{toc}{subsection}{A. NYSARC Pooled Trust (d4C) [notarized]}
\markright{A. NYSARC Pooled Trust (d4C)}

% A d4C trust was possible because the applicant was under the age of 65 and also disabled.

%%% To establish a trust: https://www.nysarctrustservices.org/wp-content/uploads/2022/01/Instructions_to_Establish_an_Account_CT_I_and_CT_II.pdf

    \phantomsection
    \addcontentsline{toc}{subsubsection}{Official check from Bankname to fund NYSARC trust}
    
    %\vspace*{2cm}
        \sectiontitle{Transfer to NYSARC Pooled Trust (d4C)}

Exempt Asset Transfer.

        %\vspace{0.5cm}
    \begin{center}
	%	\makebox[\textwidth]{\includegraphics[width=0.95\paperwidth]{check_to_NYSARC.pdf}}
[scan of official check sent to fund NYSARC]
    \end{center}
    \clearpage
    \phantomsection
    \addcontentsline{toc}{subsubsection}{NYSARC Joinder}
    \includepdf[pages=-, pagecommand={\thispagestyle{fancy}}, scale=0.9]{NYSARC Combined_Joinder_Agreement_FILLABLE.pdf}


[Note: I submitted the Joinder. I still need to send documentation of the NYSARC trust once it is created.]
    
\clearpage
    % --- FUNERAL SECTION ---
    \subsection*{B. Irrevocable Funeral Trust}
    \addcontentsline{toc}{subsection}{B. Irrevocable Funeral Trust}


    \phantomsection
    \addcontentsline{toc}{subsubsection}{Official check from Bankname to fund Funeral trust}
    \markright{B. Irrevocable Funeral Trust $\cdot$ Transfer}

   % \vspace*{2cm}
        \sectiontitle{Transfer to Irrevocable Funeral Trust}
Exempt Asset Transfer.
        %\vspace{0.5cm}
    \begin{center}
	%	\makebox[\textwidth]{\includegraphics[width=0.95\paperwidth]{check_to_FuneralHome.pdf}}
[official check to funeral home / PrePlan]
    \end{center}
    \clearpage
    \phantomsection
    \addcontentsline{toc}{subsubsection}{Funeral Trust itemization and agreement}
 \markright{B. Irrevocable Funeral Trust $\cdot$ Agreement}
     \includepdf[pages=-, pagecommand={\thispagestyle{fancy}}, scale=0.89]{PrePlan_PP-ITM-00_08.15.pdf}
     \includepdf[pages=-, pagecommand={\thispagestyle{fancy}}, scale=0.89]{PrePlan_PP-IGPECA_2020.pdf} 

% =================================================================
    % PART 5: VERIFICATION OF PAST SOURCES
    % =================================================================

  \BinderPart{PART 5}{Verification of Past Sources}
    
    \phantomsection
    \addcontentsline{toc}{subsection}{Summary of Funds Origin}
    \sectiontitle{Summary of Funds Origin}
    \noindent
    This section provides documentation for large historical deposits identified in the 5-year lookback period.
    
    \vspace{0.5cm}
    
    % ITEM A
    \noindent \textbf{A. Inheritance (\$xxxxx.xx)} \\
    \textit{Date: Month day, [year]} \\
    Proceeds from the Bankname3 Life Insurance policy of the applicant's mother, momfirstname Lastname. (See Exhibit A, p.~\pageref{page:Bankname3let}).
    
    \vspace{0.8cm} % Nice spacing between items
    
    % ITEM B
    \noindent \textbf{B. Opening Bankname Deposit (\$xxxxx.00)} \\
    \textit{Date: Month day, year} \\
    Deposit slip establishing the starting balance of the Bankname account. Represents accumulated cash savings prior to account opening. (See Exhibit B, p.~\pageref{page:openingBankname}).
    
    \vspace{0.8cm}
    
    % ITEM C
    \noindent \textbf{C. [year] Tax Returns} \\
    \textit{Date: [following year] Tax Season} 
    \begin{itemize}
        % Standard bullets align nicely under the text now
        \item \textbf{Federal (1040, p.~\pageref{page:1040}):}  Refund Issued. \\
        \textit{Note: The attached 1040 contains a calculation error which was subsequently corrected by the IRS. (See Correction Notice, p.~\pageref{page:IRScorrection}).}
        \item \textbf{NYS (IT-201, p.~\pageref{page:NYStaxes}):} Tax liability \$xxxx paid in full. \\
        \textit{Note: Liability initially paid by Daughtername Lastname. (See Part~\ref{sect:part3} A, p.~\pageref{page:reimb} for reimbursement documentation).}
    \end{itemize}

\noindent
The 1099 documents are available:
\begin{itemize}
	\item Inheritance, p.~\pageref{page:1099inheritance}
	\item SSDI, p.~\pageref{page:SS1099}
\end{itemize}
  
\noindent
\textit{Other than [the one year], taxes were not filed for recent years.}

    \vfill
    \begin{center}
        \textit{Supporting documentation follows on next pages.}
    \end{center}
    \clearpage

    %
    
    
    \phantomsection
    \addcontentsline{toc}{subsection}{A. Inheritance}

    \addcontentsline{toc}{subsubsection}{Notification letter for life insurance payout}
\markright{A. Inheritance}


\label{page:Bankname3let}
[Bankname3 inheritance letter]
  %  \includepdf[pages=-, pagecommand={\thispagestyle{fancy}}]{Bankname3_inheritance_letter.pdf}

\phantomsection
   \addcontentsline{toc}{subsubsection}{Mother momfirstname Lastname' death certificate}
\markright{A. Inheritance $\cdot$ Mother's death certificate}
\label{page:deathcert}
[death certificate to show why applicant received inheritance]
%    \includepdf[width=0.85\paperwidth, pagecommand={\thispagestyle{fancy}}]{momfirstname_Death_Cert_scan.png}

\phantomsection
    \addcontentsline{toc}{subsubsection}{1099-R}
\markright{A. Inheritance $\cdot$ 1099}
\label{page:1099inheritance}
[1099 for receiving the inheritance]
   % \includepdf[pages=-, scale=0.87, pagecommand={\thispagestyle{fancy}}]{1099inheritance.pdf} 

    \clearpage
    \phantomsection
    \addcontentsline{toc}{subsection}{B. Opening Bankname Deposit (Life savings)}
\markright{B. Opening Bankname Deposit (Life savings)}
    
\label{page:openingBankname}

[printouts from the bank showing documentation of the opening deposit]
   % \includepdf[pages=-, pagecommand={\thispagestyle{fancy}}]{openingBankname.pdf}
    
    \clearpage
    \phantomsection
    \addcontentsline{toc}{subsection}{C. Tax Returns}
	\label{sect:taxreturns}
\markright{C. Tax Returns $\cdot$ 1040 Federal}

    
    \phantomsection
    \addcontentsline{toc}{subsubsection}{1040: Federal tax return, year}

	\label{page:1040}
% trim={left lower right upper}. The clip option is necessary to ensure that the content outside the specified trim area is not displayed or accessible. 
\includepdf[pages=-,pagecommand={}, scale = 0.95, trim={1em 1em 1em 1em}, clip]{f1040.pdf}

    

% 2. The Payment (The Action)
    \newpage
    \phantomsection
    \addcontentsline{toc}{subsubsection}{Check \#xxx: Federal Tax  Payment (Tax Due)}
    \sectiontitle{IRS Tax Payment (year Tax Year)}
\markright{C. Tax Returns $\cdot$  Payment to IRS}
	\label{sec:checktoIRS}

    \begin{center}
       % \includegraphics[width=0.8\textwidth]{checktoIRSscanned.pdf}         \\
        \vspace{0.5cm}
        \textit{Note: Physical check \#xxx (\$xxxx) mailed 4/15/year; \\processed electronically by IRS on 4/xx/year.}
    \end{center}

\clearpage
    % Attach the Refund Evidence
    \phantomsection
    \addcontentsline{toc}{subsubsection}{CP12: IRS Refund / Correction Notice, tax year [year]}
	\label{page:IRScorrection}
\markright{C. Tax Returns $\cdot$ IRS correction notice}
[oops - made an arithmetic error, got a refund]
	%\includepdf[pages=-,pagecommand={}, scale = 0.95,trim={1em 1em 1em 1em}, clip]{IRScorrection.pdf} 



    % Placing the full NY tax return here as a reference document
    \phantomsection
    \addcontentsline{toc}{subsubsection}{IT-201: New York State tax return, tax year (year)}
	\label{page:NYStaxes}
\markright{C. Tax Returns $\cdot$ IT-201 New York State}

    \includepdf[pages=-, pagecommand={\thispagestyle{fancy}}, scale=0.9]{it201_2023_fill_in.pdf}





\end{document}